\documentclass[a4paper]{article}

\usepackage[slovene]{babel}
\usepackage{amsfonts,amssymb,amsmath}
\usepackage[utf8]{inputenc}
\usepackage[T1]{fontenc}
\usepackage{lmodern}
\usepackage{url}

\usepackage{graphicx}

\newtheorem{izrek}{Izrek}
\newtheorem{definicija}{Definicija}
\newtheorem{posledica}{Posledica}

\begin{document}

\title{\Huge\textbf{Hiperbolični prostori}} 
\author{\large\textsc{Tjaša Vrhovnik}\\
	Fakulteta za matematiko in fiziko\\
	Oddelek za matematiko}

\thispagestyle{empty}

\maketitle

\newpage

\tableofcontents

\newpage

%%%%%%%%%%%%%%%%%%%%%%%%%%%%%%%%%%%%%%%%%%%%%%%%%%%%%%%%%%%%%%%%%%%%%%%%%%%%
% Modeli
\section{Modeli hiperboličnih prostorov}
\subsection{Osnovne definicije}

Naslednje definicije opisujejo mnogoterosti z visoko simetrijo. Natančneje to pomeni, da so njihove grupe izometrij velike. Najpreprostejši primer so Evklidski prostori - zanje že intuitivno vemo, da obstajajo preslikave, izometrije, ki poljubni točki preslikajo eno v drugo, celo več, ortonormirano bazo v prvi točki lahko preslikajo v ortonormirano bazo v drugi. Izkazalo se bo, da to niso edini taki prostori. Drug primer so $n$-dimenzionalne sfere $\mathbb{S}^{n}$, mi pa se bomo posvetili študiju hiperboličnih prostorov.

% izometrija
\begin{definicija}
Naj bosta $(M,g)$ in $(\tilde{M}, \tilde{g})$ Riemannovi mnogoterosti. Gladka preslikava $\phi \colon M \to \tilde{M}$ \emph{ohranja metriko}, če velja $g = \phi^{*}\tilde{g}$.

Difeomorfizem, ki ohranja metriko, imenujemo \emph{izometrija}. \emph{Lokalna izometrija} pa je lokalni difeomorfizem, ki ohranja metriko. 
\end{definicija}

% frame homogeneity
\begin{definicija}
Naj bo $(M,g)$ Riemannova mnogoterost. Množico izometrij mnogoterosti $M$, ki je grupa za komponiranje, označimo z $\textup{Iso}(M,g)$.
Pravimo, da je $(M,g)$ \emph{homogena Riemannova mnogoterost}, če grupa $\textup{Iso}(M,g)$ deluje tranzitivno na $M$. To pomeni, da za poljuben par točk $p,q \in M$ obstaja izometrija $\phi \colon M \to M$ z lastnostjo $\phi(p)=q$.
\end{definicija}

Če je $\phi$ izometrija Riemannove mnogoterosti $(M,g)$, je njen diferencial $d\phi$ preslikava na tangentnem prostoru $TM$. V vsaki točki $p \in M$ diferencial definira linearno izometrijo $d\phi_{p} \colon T_{p}M \to T_{\phi(p)}M$.

\begin{definicija}
\begin{enumerate}
\item
Naj bo $p \in M$. Podgrupo grupe $\textup{Iso}(M,g)$ izometrij, ki fiksirajo $p$, imenujemo \emph{izotropična podgrupa} v $p$ in označimo z 
$\textup{Iso}_{p}(M,g)$. 
\item
Preslikavi $I_{p} \colon \textup{Iso}_{p}(M,g) \to \textup{GL}(T_{p}M)$, definirani s predpisom $I_{p}(\phi) = d\phi_{p}$, pravimo \emph{izotropična reprezentacija}.
\item
Mnogoterost $M$ je \emph{izotropična v točki $p$}, kadar izotropična reprezentacija deluje tranzitivno na množico enotskih vektorjev v $T_{p}M$. Nadalje pravimo, da je $M$ \emph{izotropična}, če je izotropična v vsaki točki $p \in M$.
\end{enumerate}
\end{definicija}

Označimo z $\textup{O}(M) = \sqcup_{p \in M} \{ \text{ortonormirane baze} \ T_{p}M \}$ množico vseh ortonormiranih baz na tangentnih prostorih mnogoterosti $M$. Delovanje grupe izometrij $\textup{Iso}(M,g)$ na množico $\textup{O}(M)$ povezuje ortonormirani bazi v točkah $p$ in $\phi(p)$ na naslednji način. Naj bo $\phi \in \textup{Iso}(M,g)$ in ${e_{1}, \dots , e_{n}} \in \textup{O}(M)$. Delovanje definiramo s predpisom
\begin{equation}\label{eq:delovanje}
\phi \cdot (e_{1}, \dots , e_{n}) = (d\phi_{p}(e_{1}), \dots , d\phi_{p}(e_{n})).
\end{equation}

\begin{definicija}
Riemannova mnogoterost $(M,g)$ je \emph{frame homogeneous} oziroma \emph{maksimalno simetrična}, če je delovanje~\ref{eq:delovanje} tranzitivno na množici $\textup{O}(M)$; natančneje, če za poljuben par $p,q \in M$ in poljuben izbor ortonormiranih baz na tangentnih prostorih $T_{p}M$ in $T_{q}M$ obstaja izometrija, ki preslika $p$ v $q$ ter ortonormirano bazo v točki $p$ v izbrano ortonormirano bazo v točki $q$.
\end{definicija}

Geometrično si homogeno Riemannovo mnogoterost predstavljamo kot tako, ki ne glede na izbor točke na njej, izgleda enako.
Izotropična Riemannova mnogoterost pa izgleda enako tudi v vseh smereh.

% konformnost
\begin{definicija}
Naj bosta $(M,g)$ in $(\tilde{M}, \tilde{g})$ Riemannovi mnogoterosti. Difeomorfizem $\phi \colon M \to \tilde{M}$ je \emph{konformna preslikava}, če obstaja taka pozitivna funkcija $\mu \in \mathcal{C}^{\infty}(M)$, da velja
\[ \phi^{*}\tilde{g} = \mu g. \]
V tem primeru pravimo, da sta mnogoterosti $(M,g)$ in $(\tilde{M}, \tilde{g})$ \emph{konformno ekvivalentni}.
\end{definicija}

Konformni difeomorfizmi med Riemannovimi mnogoterostmi so ravno difeomorfizmi, ki ohranjajo velikosti kotov. Tako se pomen zgornje definicije sklada s konformnostjo, ki jo poznamo iz kompleksne analize.

Posebej zanimive Riemannove mnogoterosti so tiste, ki jih (vsaj) lokalno lahko primerjamo z Evklidskim prostorom. Pravimo, da je Riemannova mnogoterost $(M,g)$ \emph{lokalno konformno ploska}, če ima vsaka točka $p\in M$ okolico, ki je konformno ekvivalentna odprti množici v $(\mathbb{R}^{n}, \bar{g})$, kjer $\bar{g}$ označuje običajno Evklidsko metriko.

%%%%%%%%
% Modeli
\subsection{Modeli}
V tem razdelku bomo navedli modele hiperboličnih prostorov, ki so "frame homogeneous" Riemannove mnogoterosti dimenzije $n \geq 1$. Sprva jih bomo le navedli, kasneje pa pokazali njihovo frame homogeneity. Izkaže se, da so vsi ti modeli med seboj izometrični, zato lahko v praksi izberemo kateregakoli izmed njih, na njem obravnavamo želeno in to prenesemo na splošen hiperbolični prostor te dimenzije.

Naj bo $n \geq 1$ in izberimo $R>0$.
\emph{$(n+1)$-dimenzionalni prostor Minkowskega} je prostor $\mathbb{R}^{n,1}$, ki ga v standardnih koordinatah $(x^{1}, \dots , x^{n}, \tau)$ opremimo z \emph{metriko Minkowskega}
\begin{equation}
\bar{q}^{n,1} = (dx^{1})^2 + \cdots + (dx^{n})^2 - (d\tau)^2.
\end{equation}
Metriko $\bar{q}^{n,1}$ bomo v nadaljevanju označevali preprosto s $\bar{q}$.

\begin{enumerate}
\item
% hiperboloid
\textsc{Hiperboloid} $\mathbb{H}^{n}(R)$ definramo na naslednji način.

Vzemimo $(n+1)$-dimenzionalni prostor Minkowskega $\mathbb{R}^{n,1}$ s standardnimi koordinatami $(x^{1}, \dots , x^{n}, \tau)$ in metriko $\bar{q}$.
Pozitivni del ($\tau > 0$) dvodelnega hiperboloida $(x^{1})^2 + \cdots + (x^{n})^2 - \tau^2 = -R^2$ opremimo z metriko
\begin{equation}\label{eq:H^n metrika}
\u g_{R}^{1} = \iota^{*}\bar{q},
\end{equation}
kjer $\iota$ označuje inkluzijo $\iota \colon \mathbb{H}^{n}(R) \to \mathbb{R}^{n,1}$. Dobljeno podmnogoterost $(\mathbb{H}^{n}(R), \u g_{R}^{1})$ imenujemo \emph{hiperboloid} dimenzije $n$ s polmerom $R$.

\item
% Beltrami-Klein
\textsc{Beltrami-Kleinov model} $\mathbb{K}^{n}(R)$

Na $n$-dimenzionalni krogli $\mathbb{K}^{n}(R)$ s središčem v izhodišču prostora $\mathbb{R}^{n}$ in polmerom $R$ uvedimo koordinate $(w^{1}, \dots , w^{n})$. Kroglo opremimo z metriko
\begin{equation}\label{eq:K^n metrika}
\u g_{R}^{2} = R^2 \frac{(dw^{1})^2 + \cdots + (dw^{n})^2}{R^2-|w|^2} + R^2 \frac{(w^{1}dw^{1} + \cdots + w^{n}dw^{n})^2}{(R^2-|w|^2)^2}.
\end{equation}
Mnogoterost $(\mathbb{K}^{n}(R), \u g_{R}^{2})$ se imenuje \emph{Beltrami-Kleinov model}.

\item
% Poincarejeva korgla
\textsc{Poincar\'ejeva krogla} $\mathbb{B}^{n}(R)$

Na $n$-dimenzionalni krogli $\mathbb{B}^{n}(R)$ s središčem v izhodišču prostora $\mathbb{R}^{n}$ in polmerom $R$ uvedimo koordinate $(u^{1}, \dots , u^{n})$. Kroglo opremimo z metriko
\begin{equation}\label{eq:K^n metrika}
\u g_{R}^{3} = 4R^4 \frac{(du^{1})^2 + \cdots + (du^{n})^2}{(R^2-|u|^2)^2}.
\end{equation}
Mnogoterost $(\mathbb{B}^{n}(R), \u g_{R}^{3})$ se imenuje \emph{Poincar\'ejeva krogla}.

\item
% Poincarejev polprostor
\textsc{Poincar\'ejev polprostor} $\mathbb{U}^{n}(R)$

Na Evklidskem prostoru $\mathbb{R}^{n}$ uvedimo koordinate $(x^{1}, \dots , x^{n-1}, y)$ in njegov podprostor $\mathbb{U}^{n}(R) = \{ (x,y); \ y>0 \}$ opremimo z metriko
\begin{equation}\label{eq:K^n metrika}
\u g_{R}^{4} = R^2 \frac{(dx^{1})^2 + \cdots + (dx^{n-1})^2 + dy^2}{y^2}.
\end{equation}
Mnogoterosti $(\mathbb{U}^{n}(R), \u g_{R}^{4})$ pravimo \emph{Poincar\'ejev polprostor}.

%
\end{enumerate}

%%%%%%%%%%%%%%%%%%%%%%%%%%%%%%%%%%%%%%%%%%%%%%%%%%%%%%%%%%%%%%%%%%%%%%%%%%%%
% Hiperbolična geometrija
\section{Hiperbolična geometrija}

%%%%%%%%%%%%%%%%%%%%%%%%%%%%%%%%%%%%%%%%%%%%%%%%%%%%%%%%%%%%%%%%%%%%%%%%%%%%
% Literatura

\begin{thebibliography}{99}

\end{thebibliography}

\end{document}