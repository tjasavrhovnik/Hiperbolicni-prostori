\documentclass[a4paper]{article}

\usepackage[slovene]{babel}
\usepackage{amsfonts,amssymb,amsmath}
\usepackage[utf8]{inputenc}
\usepackage[T1]{fontenc}
\usepackage{lmodern}
\usepackage{url}

\usepackage{graphicx}

\newtheorem{izrek}{Izrek}
\newtheorem{definicija}{Definicija}
\newtheorem{trditev}{Trditev}
\newtheorem{posledica}{Posledica}
\newtheorem{primer}{Primer}

\begin{document}

\title{\Huge\textbf{Hiperbolični prostori}} 
\author{\large\textsc{Tjaša Vrhovnik}\\
	Fakulteta za matematiko in fiziko\\
	Oddelek za matematiko}

\thispagestyle{empty}

\maketitle

\newpage

\tableofcontents

\newpage

%%%%%%%%%%%%%%%%%%%%%%%%%%%%%%%%%%%%%%%%%%%%%%%%%%%%%%%%%%%%%%%%%%%%%%%%%%%%
% Modeli
\section{Modeli hiperboličnih prostorov}
%%%%%%%%%%%%%%%%%%%
% Osnovne definicije
\subsection{Osnovne definicije}

Naslednje definicije opisujejo mnogoterosti z visoko simetrijo. Natančneje to pomeni, da so njihove grupe izometrij velike. Najpreprostejši primer so Evklidski prostori - zanje že intuitivno vemo, da obstajajo preslikave, izometrije, ki poljubni točki preslikajo eno v drugo, celo več, ortonormirano bazo v prvi točki lahko preslikajo v ortonormirano bazo v drugi. Izkazalo se bo, da to niso edini taki prostori. Drug primer so $n$-dimenzionalne sfere $\mathbb{S}^{n}$, mi pa se bomo posvetili študiju hiperboličnih prostorov.

% izometrija
\begin{definicija}
Naj bosta $(M,g)$ in $(\tilde{M}, \tilde{g})$ Riemannovi mnogoterosti. Gladka preslikava $\phi \colon M \to \tilde{M}$ \emph{ohranja metriko}, če velja $g = \phi^{*}\tilde{g}$.

Difeomorfizem, ki ohranja metriko, imenujemo \emph{izometrija}. \emph{Lokalna izometrija} pa je lokalni difeomorfizem, ki ohranja metriko. 
\end{definicija}

% frame homogeneity
\begin{definicija}
Naj bo $(M,g)$ Riemannova mnogoterost. Množico izometrij mnogoterosti $M$, ki je grupa za komponiranje, označimo z $\textup{Iso}(M,g)$.
Pravimo, da je $(M,g)$ \emph{homogena Riemannova mnogoterost}, če grupa $\textup{Iso}(M,g)$ deluje tranzitivno na $M$. To pomeni, da za poljuben par točk $p,q \in M$ obstaja izometrija $\phi \colon M \to M$ z lastnostjo $\phi(p)=q$.
\end{definicija}

Če je $\phi$ izometrija Riemannove mnogoterosti $(M,g)$, je njen diferencial $d\phi$ preslikava na tangentnem prostoru $TM$. V vsaki točki $p \in M$ diferencial definira linearno izometrijo $d\phi_{p} \colon T_{p}M \to T_{\phi(p)}M$.

\begin{definicija}
Naj bo $p \in M$. Podgrupo grupe $\textup{Iso}(M,g)$ izometrij, ki fiksirajo $p$, imenujemo \emph{izotropična podgrupa} v $p$ in označimo z 
$\textup{Iso}_{p}(M,g)$. 

Preslikavi $I_{p} \colon \textup{Iso}_{p}(M,g) \to \textup{GL}(T_{p}M)$, definirani s predpisom $I_{p}(\phi) = d\phi_{p}$, pravimo \emph{izotropična reprezentacija}.

Mnogoterost $M$ je \emph{izotropična v točki $p$}, kadar izotropična reprezentacija deluje tranzitivno na množico enotskih vektorjev v $T_{p}M$. Nadalje pravimo, da je $M$ \emph{izotropična}, če je izotropična v vsaki točki $p \in M$.
\end{definicija}

Označimo z $\textup{O}(M) = \sqcup_{p \in M} \{ \text{ortonormirane baze} \ T_{p}M \}$ množico vseh ortonormiranih baz na tangentnih prostorih mnogoterosti $M$. Delovanje grupe izometrij $\textup{Iso}(M,g)$ na množico $\textup{O}(M)$ povezuje ortonormirani bazi v točkah $p$ in $\phi(p)$ na naslednji način. Naj bo $\phi \in \textup{Iso}(M,g)$ in $\{e_{1}, \dots , e_{n} \} \in \textup{O}(M)$. Delovanje definiramo s predpisom
\begin{equation}\label{eq:delovanje}
\phi \cdot (e_{1}, \dots , e_{n}) = (d\phi_{p}(e_{1}), \dots , d\phi_{p}(e_{n})).
\end{equation}

\begin{definicija}
Riemannova mnogoterost $(M,g)$ je \emph{frame homogeneous} oziroma \emph{maksimalno simetrična}, če je delovanje~\ref{eq:delovanje} tranzitivno na množici $\textup{O}(M)$; natančneje, če za poljuben par $p,q \in M$ in poljuben izbor ortonormiranih baz na tangentnih prostorih $T_{p}M$ in $T_{q}M$ obstaja izometrija, ki preslika $p$ v $q$ ter ortonormirano bazo v točki $p$ v izbrano ortonormirano bazo v točki $q$.
\end{definicija}

Geometrično si homogeno Riemannovo mnogoterost predstavljamo kot tako, ki v vsaki točki na njej izgleda enako.
Izotropična Riemannova mnogoterost pa izgleda enako tudi v vseh smereh.

% konformnost
\begin{definicija}
Naj bosta $(M,g)$ in $(\tilde{M}, \tilde{g})$ Riemannovi mnogoterosti. Difeomorfizem $\phi \colon M \to \tilde{M}$ je \emph{konformna preslikava}, če obstaja taka pozitivna funkcija $\mu \in \mathcal{C}^{\infty}(M)$, da velja
\[ \phi^{*}\tilde{g} = \mu g. \]
V tem primeru pravimo, da sta mnogoterosti $(M,g)$ in $(\tilde{M}, \tilde{g})$ \emph{konformno ekvivalentni}.
\end{definicija}

Konformni difeomorfizmi med Riemannovimi mnogoterostmi so ravno difeomorfizmi, ki ohranjajo velikosti kotov. Tako se pomen zgornje definicije sklada s konformnostjo, ki jo poznamo iz kompleksne analize.

Posebej zanimive Riemannove mnogoterosti so tiste, ki jih (vsaj) lokalno lahko primerjamo z Evklidskim prostorom. Pravimo, da je Riemannova mnogoterost $(M,g)$ \emph{lokalno konformno ploska}, če ima vsaka točka $p\in M$ okolico, ki je konformno ekvivalentna odprti množici v $(\mathbb{R}^{n}, \bar{g})$, kjer $\bar{g}$ označuje običajno Evklidsko metriko. Videli bomo, da imajo hiperbolični prostori to lastnost.

%%%%%%%%%%%%%%%%%%%
% Modeli
\subsection{Modeli}

V tem razdelku bomo navedli modele hiperboličnih prostorov, ki so "frame homogeneous" Riemannove mnogoterosti dimenzije $n \geq 1$. Sprva jih bomo le navedli, kasneje pa pokazali njihovo frame homogeneity. Izkaže se, da so vsi ti modeli med seboj izometrični, zato lahko v praksi izberemo kateregakoli izmed njih, na njem obravnavamo želeno in to prenesemo na splošen hiperbolični prostor te dimenzije.

Naj bo $n \geq 1$ in izberimo $R>0$.
\emph{$(n+1)$-dimenzionalni prostor Minkowskega} je prostor $\mathbb{R}^{n,1}$, ki ga v standardnih koordinatah $(x^{1}, \dots , x^{n}, \tau)$ opremimo z \emph{metriko Minkowskega}
\begin{equation}\label{eq:Mink metrika}
\bar{q}^{n,1} = (dx^{1})^2 + \cdots + (dx^{n})^2 - (d\tau)^2.
\end{equation}
Metriko $\bar{q}^{n,1}$ bomo v nadaljevanju označevali preprosto s $\bar{q}$.

\begin{primer}
4-dimenzionalni prostor Minkowskega $\mathbb{R}^{3,1}$ s koordinatami $(x,y,z,t)$ opisuje prostor-čas v Einsteinovi teoriji relativnosti. Grupo izometrij, $\textup{O}(3,1)$, imenovano \emph{Poncar\'ejeva grupa} sestavljajo $10$ generatorjev: tri prostorske in ena časovna translacija, tri rotacije (v ravninah $(x,y)$, $(x,z)$, $(y,z)$) in tri "time boosts" (rotacije v ravninah $(t,x)$, $(t,y)$, $(t,z)$).
\end{primer}

Sedaj definirajmo štiri Riemannove mnogoterosti, ki so osnovni modeli hiperboličnega prostora dimenzije $n$ in polmera $R$.

%%%%%%%%%%%
\begin{enumerate}
\item
% hiperboloid
\textsc{Hiperboloid} $\mathbb{H}^{n}(R)$.

Vzemimo $(n+1)$-dimenzionalni prostor Minkowskega $\mathbb{R}^{n,1}$ s standardnimi koordinatami $(x^{1}, \dots , x^{n}, \tau)$ in metriko $\bar{q}$.
Pozitivni del ($\tau > 0$) dvodelnega hiperboloida $(x^{1})^2 + \cdots + (x^{n})^2 - \tau^2 = -R^2$ opremimo z metriko
\begin{equation}\label{eq:H^n metrika}
\u g_{R}^{1} = \iota^{*}\bar{q},
\end{equation}
kjer $\iota$ označuje inkluzijo $\iota \colon \mathbb{H}^{n}(R) \to \mathbb{R}^{n,1}$. Dobljeno podmnogoterost $(\mathbb{H}^{n}(R), \u g_{R}^{1})$ imenujemo \emph{hiperboloid} dimenzije $n$ s polmerom $R$.

\item
% Beltrami-Klein
\textsc{Beltrami-Kleinov model} $\mathbb{K}^{n}(R)$

Na $n$-dimenzionalni krogli $\mathbb{K}^{n}(R)$ s središčem v izhodišču prostora $\mathbb{R}^{n}$ in polmerom $R$ uvedimo koordinate $(w^{1}, \dots , w^{n})$. Kroglo opremimo z metriko
\begin{equation}\label{eq:K^n metrika}
\u g_{R}^{2} = R^2 \frac{(dw^{1})^2 + \cdots + (dw^{n})^2}{R^2-|w|^2} + R^2 \frac{(w^{1}dw^{1} + \cdots + w^{n}dw^{n})^2}{(R^2-|w|^2)^2}.
\end{equation}
Mnogoterost $(\mathbb{K}^{n}(R), \u g_{R}^{2})$ se imenuje \emph{Beltrami-Kleinov model}.

\item
% Poincarejeva korgla
\textsc{Poincar\'ejeva krogla} $\mathbb{B}^{n}(R)$

Na $n$-dimenzionalni krogli $\mathbb{B}^{n}(R)$ s središčem v izhodišču prostora $\mathbb{R}^{n}$ in polmerom $R$ uvedimo koordinate $(u^{1}, \dots , u^{n})$. Kroglo opremimo z metriko
\begin{equation}\label{eq:K^n metrika}
\u g_{R}^{3} = 4R^4 \frac{(du^{1})^2 + \cdots + (du^{n})^2}{(R^2-|u|^2)^2}.
\end{equation}
Mnogoterost $(\mathbb{B}^{n}(R), \u g_{R}^{3})$ definira \emph{Poincar\'ejevo kroglo}.

\item
% Poincarejev polprostor
\textsc{Poincar\'ejev polprostor} $\mathbb{U}^{n}(R)$

Na Evklidskem prostoru $\mathbb{R}^{n}$ uvedimo koordinate $(x^{1}, \dots , x^{n-1}, y)$ in njegov podprostor $\mathbb{U}^{n}(R) = \{ (x,y); \ y>0 \}$ opremimo z metriko
\begin{equation}\label{eq:K^n metrika}
\u g_{R}^{4} = R^2 \frac{(dx^{1})^2 + \cdots + (dx^{n-1})^2 + dy^2}{y^2}.
\end{equation}
Mnogoterosti $(\mathbb{U}^{n}(R), \u g_{R}^{4})$ pravimo \emph{Poincar\'ejev polprostor}.
%
\end{enumerate}
%%%%%%%%%%

Zaradi izometričnosti zgornjih modelov pogosto hiperbolični prostor dimenzije $n$ s polmerom $R$ označimo z $\mathbb{H}^{n}(R)$, metriko pa z $\u g_{R}$, pri čemer imamo v mislih poljubnega izmed modelov. Če za polmer izberemo $R=1$, Riemannovo mnogoterost označimo z $(\mathbb{H}^{n}, \u g)$ in imenujemo \emph{hiperbolični prostor} dimenzije $n$. V dveh dimenzijah dobimo \emph{hiperbolično ravnino}, h kateri se bomo vrnili kasneje.

%%%%%%%%%%%%%%%%%%%
% Izometričnost modelov
\subsection{Izometričnost modelov}

\begin{izrek}
Modeli $n$-dimenzionalnih hiperboličnih prostorov s polmerom $R$ so paroma izometrični.
\end{izrek}

Dokaz bo potekal v več korakih. Najprej moramo preveriti, da je hiperbolični prostor Riemannova mnogoterost. Nato bomo konstruirali izometrije med naslednjimi pari modelov: hiperboloidom in Beltrami-Kleinovim modelom, hiperboloidom in Poincar\'ejevo kroglo ter Poincar\'ejevim polprostorom in Poincar\'ejevo kroglo.

Dokažimo najprej, da je hiperboloid $\mathbb{H}^{n}(R)$ Riemannova podmnogoterost prostora $\mathbb{R}^{n,1}$. To zadostuje, saj bo iz izometričnosti sledilo, da so vsi modeli Riemannove mnogoterosti.
Prostor $\mathbb{H}^{n}(R)$ lahko opišemo kot 
\begin{equation}\label{eq:H^n enacba}
\mathbb{H}^{n}(R) = F^{-1}(-R^2) \cap \{ \tau>0 \},
\end{equation}
kjer je preslikava $F \colon \mathbb{R}^{n+1} \to \mathbb{R}$ definirana s predpisom $F(x^{1}, \dots , x^{n}, \tau) = \sum_{i=1}^{n} (x^{i})^2 - \tau^2$.
Tangentni prostor v točki $p \in \mathbb{H}^{n}(R)$, ki je enak $T_{p}\mathbb{H}^{n}(R) = \ker dF_{p}$, razpenjajo vektorji $X^{i} = \tau \frac{\partial}{\partial x^{i}} + x^{i} \frac{\partial}{\partial \tau}$ za $i = 1, \dots , n$. Ker je njihov produkt (glede na metriko $\bar{q}$) pozitiven, je metrika $\u g_{R}^{1}$ pozitivno definitna in določa Riemannovo podmnogoterost $(\mathbb{H}^{n}(R), \u g_{R}^{1})$. 

\subsubsection{Centralna projekcija}
Izometrija med hiperboloidom in Beltrami-Kleinovim modelom se imenuje \emph{centralna projekcija}, $c \colon \mathbb{H}^{n}(R) \to \mathbb{K}^{n}(R)$.

Naj bo $T=(x^{1}, \dots , x^{n}, \tau) \in \mathbb{H}^{n}(R)$ poljubna točka. Presečišče premice $OT$ skozi izhodišče v $\mathbb{R}^{n,1}$ in točko $T$ s hiperravnino $\{ (x, \tau); \ \tau=R \}$ označimo z $Y \in \mathbb{R}^{n,1}$. Pišimo $Y=(y,R)$, kjer je $y = c(T) \in \mathbb{K}^{n}(R)$ slika $T$ s centralno projekcijo. 
Tedaj se koordinate točke $Y$ izražajo s koordinatami $T$, natančneje, $Y=\frac{R}{\tau} T$. Preslikavo $c$ zato podaja zveza
\begin{equation}\label{eq: cent-proj}
c(x, \tau) = \frac{R}{\tau} x.
\end{equation}
Ker je slika s preslikavo $c$ točka na hiperboloidu, velja
\[ |c(x,\tau)|^2 = \frac{R^2}{\tau^2} |x|^2 = \frac{R^2}{\tau^2}(\tau^2-R^2) = R^2 \left(1-\frac{R^2}{\tau^2} \right) < R^2, \]
torej je $c(x, \tau) \in \mathbb{K}^{n}(R)$ in centralna projekcija je dobro definirana.

Da je preslikava $c$ difeomorfizem, bomo pokazali s konstrukcijo njenega inverza.
Vzemimo poljubno točko $y \in \mathbb{K}^{n}(R)$. Potem obstaja enoličen $\lambda>0$, da velja $(x, \tau) = \lambda (y,R) \in \mathbb{H}^{n}(R)$. Slednja točka je določena z zvezo $\lambda^2 (|y|^2-R^2)=-R^2$, od koder sledi $\lambda = \frac{R}{(R^2-|y|^2)^{1/2}}$.
Potem predpis
\begin{equation}\label{eq:c^{-1}}
c^{-1}(y) = \left( \frac{Ry}{(R^2-|y|^2)^{1/2}}, \frac{R^2}{(R^2-|y|^2)^{1/2}} \right)
\end{equation}
definira gladko preslikavo, ki je inverz centralne projekcije.

Nazadnje dokažimo, da je $c$ izometrija. Želimo videti, da velja enakost $(c^{-1})^{*} \u g_{R}^{1} = \u g_{R}^{2}$.
Diferenciala točke $(x, \tau) = c^{-1}(y)$ iz enačbe~\ref{eq:c^{-1}} za nek $y \in \mathbb{K}^{n}(R)$ sta enaka
\[ dx^{i} = \frac{Rdy^{k}}{(R^2-|y|^2)^{1/2}} + \frac{Ry^{i} \sum_{k=1}^{n}y^{k}dy^{k}}{(R^2-|y|^2)^{3/2}}, \]
\[ d\tau = \frac{R^2 \sum_{k=1}^{n}y^{k}dy^{k}}{(R^2-|y|^2)^{3/2}}. \]
Od tod izračunamo
\[ (c^{-1})^{*} \u g_{R}^{1} = \sum_{i=1}^{n} (dx^{i})^2 - (d\tau)^2 = R^2 \frac{\sum_{i=1}^{n}(dy^{i})^2}{R^2-|y|^2} + R^2 \frac{(\sum_{i=1}^{n}y^{i}dy^{i})^2}{(R^2-|y|^2)^2} = \u g_{R}^{2}. \]
Torej je centralna projekcija res izometrija med $(\mathbb{H}^{n}(R), \u g_{R}^{1})$ in $(\mathbb{K}^{n}(R), \u g_{R}^{2})$.

\subsubsection{Hiperbolična stereografska projekcija}

\subsubsection{Posplošena Cayleyjeva transformacija}

%%%%%%%%%%%%%%%%%%%
% Lastnosti hiperboličnih prostorov
\subsection{Lastnosti hiperboličnih prostorov}

\begin{trditev}
Hiperbolični prostor $(\mathbb{H}^{n}(R), \u g)$ je lokalno konformno ploščat.
\end{trditev}

Dokaz:
Za model hiperboličnega prostora vzemimo Poincar\'ejevo kroglo  $\mathbb{B}^{n}(R)$. Identična preslikava $id \colon (\mathbb{B}^{n}(R), \u g_{R}^{3}) \to (\mathbb{R}^{n}, \bar{g})$ je konformna in porodi konformno ekvivalenco Poincar\'ejeve krogle z odprto podmnožico Evklidskega prostora. Zaradi izometričnosti modelov sledi, da je hiperbolični prostor lokalno konformno ploščat.

% Lorentzova grupa
\begin{definicija}
Naj bo $\mathbb{R}^{n,1}$ prostor Minkowskega ($n \geq 1$) opremljen z metriko Minkowskega $\bar{q}$. Grupo linearnih preslikav, ki slikajo $\mathbb{R}^{n,1}$ vase in ohranjajo metriko Minkowskega, imenujemo \emph{($n+1$)-dimenzionalna Lorentzova grupa} in označimo z $\textup{O}(n,1)$.

Njeno podgrupo, ki ohranja hiperboloid $\mathbb{H}^{n}(R)$ označimo z $\textup{O}^{+}(n,1)$. Pravimo ji \emph{ortokrona Lorentzova grupa}.
\end{definicija}

\begin{trditev}
Lorentzova grupa $\textup{O}^{+}(n,1)$ deluje tranzitivno na množico $\textup{O}(\mathbb{H}^{n}(R))$. Hiperbolični prostor $\mathbb{H}^{n}(R)$ je "frame homogeneous".
\end{trditev}

Dokaz:
Dovolj je pokazati, da za poljubno točko $p \in \mathbb{H}^{n}(R)$ in ortonormirano bazo $\{e_{1}, \dots , e_{n} \}$ prostora $T_{p}\mathbb{H}^{n}(R)$ obstaja ortogonalna preslikava, ki preslika točko $N=(0, \dots , 0, R)$ v $p$ ter ortonormirano bazo $\{\partial_{1}, \dots , \partial_{n} \}$ za $T_{N}\mathbb{H}^{n}(R)$ v bazo $\{e_{i}\}_{i}$.

Naj bo $p \in \mathbb{H}^{n}(R)$ poljubna točka. Identificirajmo $p$ z vektorjem dolžine $R$ v $T_{p}\mathbb{R}^{n,1}$  in postavimo $\bar{p} = \frac{p}{R}$. $\bar{p}$ je enotski vektor glede na metriko Minkowskega $\bar{q}$, kaže v smeri $p$, in $\bar{p} \in T_{p}\mathbb{R}^{n,1}$.
Prostor $\mathbb{H}^{n}(R)$ predstavimo kot v~\ref{eq:H^n enacba}.
Tedaj je gradient preslikave $F$ glede na metriko $\bar{q}$ enak
\begin{equation}
\textup{grad} F = 2 \sum_{i=1}^{n} x^{i} \frac{\partial}{\partial x^{i}} + 2\tau \frac{\partial}{\partial \tau}. \nonumber
\end{equation} 
Opazimo, da je $\bar{p} = 2 \cdot \textup{grad} F$ in zato $\bar{p} \perp T_{p}\mathbb{H}^{n}(R)$.
Potem je množica $\{ e_{1}, \dots , e_{n}, \bar{p} \}$ ortonormirana baza prostora $\mathbb{R}^{n,1}$ glede na metriko $\bar{q}$.

Naj bo $M$ matrika, katere stolpci so vektorji iz zgornje ortonormirane baze. Po konstrukciji velja $M \in \textup{O}^{+}(n,1)$ in $M$ slika $\{\partial_{1}, \dots , \partial_{n+1} \}$ v $\{e_{1}, \dots , e_{n}, \bar{p} \}$ ter $N$ v $p$.
Ker je $M$ kot preslikava linearna na $\mathbb{R}^{n,1}$, je njen diferencial v točki $N$, $dM_{N} \colon T_{N}\mathbb{R}^{n,1} \to T_{p}\mathbb{R}^{n,1}$, prav tako predstavljen z matriko $M$. To pa pomeni, da je $dM_{n}(\partial_{i}) = e_{i}$, $i = 1, \dots , n$.
$M$ je iskana preslikava in delovanje grupe je tranzitivno. Po definiciji je zato $\mathbb{H}^{n}(R)$ frame homogeneous.

%%%%%%%%%%%%%%%%%%%%%%%%%%%%%%%%%%%%%%%%%%%%%%%%%%%%%%%%%%%%%%%%%%%%%%%%%%%%
% Hiperbolična geometrija
\section{Hiperbolična geometrija}

%%%%%%%%%%%%%%%%%%%%%%%%%%%%%%%%%%%%%%%%%%%%%%%%%%%%%%%%%%%%%%%%%%%%%%%%%%%%
% Literatura

\begin{thebibliography}{99}

\end{thebibliography}

\end{document}